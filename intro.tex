\ssr{ВВЕДЕНИЕ}

Проблема некорректных заимствований в научной и образовательной среде имеет многоуровневый характер и включает не только явный плагиат, но и более скрытые формы нарушения академической этики: ошибочные ссылки, неправильно приписанные источники, искаженные цитаты, отсутствие достоверной библиографии. Масштаб проблемы растет с увеличением объема научных работ и доступности технологий, позволяющих легко манипулировать текстами.

Проблема обнаружения нечетких дубликатов является одной из наиболее важных и трудных задач анализа веб-данных и поиска информации. Основным препятствием для успешного решения данной задачи является гигантский объем данных, что делает практически невозможным попарное сравнение текстов документов в разумное время~\cite{zelenkov2007}.

Современные подходы к выявлению некорректных заимствований выходят далеко за рамки простого поиска текстовых совпадений. Требуется применение комплекса методов обработки естественного языка (NLP), машинного обучения, информационного поиска (IR) и автоматизированных систем верификации источников. Каждый класс алгоритмов решает специфическую задачу: от синтаксического обнаружения копируемых фрагментов до семантического анализа парафраза и проверки корректности ссылок на литературу~\cite{cyberleninka_zaimstvovaniya}.

Цель данной работы --- классифицировать и описать основные методы и алгоритмы, применяемые для выявления некорректных заимствований.

\clearpage

\chapter{Классификация и основные подходы к обнаружению заимствований}

\section{Классификация алгоритмов и методов детекции}

Все многообразие подходов к обнаружению некорректных заимствований можно систематизировать в пять основных классов:

\begin{enumerate}

\item методы синтаксического сравнения и <<отпечатков>> текста;

\item лексико-статистические методы и метрики схожести;

\item методы семантического/смыслового сопоставления;

\item модели авторского стиля и стилометрия;

\item алгоритмы анализа и верификации цитирования.

\end{enumerate}

Такая классификация позволяет проанализировать каждый подход с точки зрения его особенностей, преимуществ и ограничений~\cite{zelenkov2007}.

\chapter{Синтаксические и лексико-статистические методы}

\section{Методы синтаксического сравнения и отпечатков текста}

\subsection{Общие принципы}

Для решения задачи обнаружения нечетких дубликатов текстов применяются методы синтаксического сравнения. Идея этих методов заключается в том, чтобы получить компактное представление текста, сохраняющее его уникальные черты, и сравнивать эти представления вместо полных текстов. Такое представление называется <<отпечатком>> или сигнатурой документа. Это позволяет значительно ускорить сравнение больших объемов текстовых данных. Основной принцип состоит в следующем: текст разбивается на перекрывающиеся или неперекрывающиеся фрагменты фиксированной длины, каждому фрагменту вычисляется хеш-значение, и затем сравниваются хеши~\cite{zelenkov2007}.

\subsection{Шинглы и метод Бродера}

Одним из первых исследований в области нахождения нечетких дубликатов является работа А. Бродера, в которой был предложен синтаксический метод оценки сходства между документами, основанный на представлении документа в виде множества всевозможных последовательностей фиксированной длины $k$, состоящих из соседних слов. Такие последовательности были названы шинглами (k-грамм слов). Два документа считались похожими, если их множества шинглов существенно пересекались.

Поскольку число шинглов примерно равно длине документа в словах, были предложены два метода сэмплирования для получения репрезентативных подмножеств. Первый метод оставлял только те шинглы, чьи дактилограммы (численные отпечатки, вычисляемые по алгоритму Рабина-Карпа) делились без остатка на некоторое число $m$. Второй метод отбирал фиксированное число $s$ шинглов с наименьшими значениями дактилограмм~\cite{zelenkov2007}.

\subsection{Алгоритм Winnowing}

Winnowing --- это модификация простого $k$-граммного анализа, предложенная для повышения эффективности. Алгоритм работает следующим образом: текст разбивается на $k$-граммы (подстроки из $k$ символов); для каждой $k$-граммы вычисляется хеш-значение; из всех хешей выбираются только минимальные значения в пределах скользящего окна размером $w$. Эти избранные хеши образуют отпечаток документа, после чего отпечатки различных документов сравниваются для определения схожести. Преимущество алгоритма Winnowing состоит в устойчивости к перестановкам фрагментов и синтаксическим изменениям, при этом сохраняя низкую вычислительную сложность. Алгоритм используется в системах MOSS и JPlag для обнаружения плагиата в исходном коде и текстах~\cite{matec_winnowing2018}.

\subsection{Дактилограммы и алгоритм Рабина-Карпа}

Одними из первых исследований в области нахождения нечетких дубликатов являются работы U. Manber и N. Heintze. Дактилограмма (также называемая отпечатком или хеш-сигнатурой) файла или документа включает все текстовые подстроки фиксированной длины. Численное значение дактилограмм вычисляется с помощью алгоритма случайных полиномов Рабина-Карпа. Дактилограмма отличается от простого отпечатка тем, что представляет собой набор множественных хеш-значений для разных подстрок, а не единственное значение. В качестве меры сходства двух документов используется отношение числа общих подстрок к размеру файла или документа. Алгоритм Рабина-Карпа основан на быстром вычислении хешей для перекрывающихся подстрок с использованием скользящего окна. Полиномиальное хеширование позволяет за $O(1)$ пересчитать хеш для следующей подстроки на основе хеша предыдущей. Метод особенно эффективен при поиске точных совпадений в больших массивах текста~\cite{zelenkov2007}.

\subsection{Мегашинглы и супершинглы}

Дальнейшим развитием концепций Бродера являются исследования D. Fetterly. Для каждого документа вычисляются 84 дактилограммы по алгоритму Рабина-Карпа с помощью взаимно-однозначных и независимых функций. В результате каждый документ представлялся 84 шинглами, минимизирующими значение соответствующей функции. Затем 84 шингла разбиваются на 6 групп по 14 шинглов в каждой. Эти группы называются супершинглами. Документ представляется всевозможными попарными сочетаниями из 6 супершинглов, которые называются мегашинглами. Число таких мегашинглов равно 15. Два документа сходны по содержанию, если у них совпадает хотя бы один мегашингл. Ключевое преимущество данного алгоритма состоит в том, что любой документ (в том числе и очень маленький) всегда представляется вектором фиксированной длины, и сходство определяется простым сравнением координат вектора~\cite{zelenkov2007}.

\subsection{N-граммный анализ и SimHash}

$N$-граммный анализ --- одна из самых базовых, но действенных методик. Документ представляется как набор $n$-грамм (последовательности из $n$ символов или слов). Сравнение документов производится по пересечению их $n$-грамм. Алгоритм не учитывает порядок $n$-грамм в документе и работает с ними как с множеством. SimHash расширяет эту идею: для каждого документа строится битовый отпечаток фиксированной длины путем комбинирования хешей его $n$-грамм. Два документа считаются схожими, если расстояние Хемминга между их отпечатками мало~\cite{mdpi_fingerprint2021}.

\subsection{Locality Sensitive Hashing (LSH)}

LSH --- это техника быстрого поиска похожих текстов в больших коллекциях. Алгоритм хеширует входные элементы таким образом, что похожие элементы с высокой вероятностью получают один и тот же хеш или хеши в одних и тех же <<корзинах>>. Для поиска похожих текстов на основе LSH создается несколько таблиц хешей с разными хеш-функциями. Входной документ хешируется со всеми этими функциями, и его представление проверяется в каждой таблице. Документы, попадающие в одну и ту же корзину, являются кандидатами на сходство. Это позволяет за логарифмическое или даже суб-логарифмическое время находить кандидатов на сходство без полного сравнения со всеми документами в базе. Метод особенно ценен для масштабируемых систем с миллионами или миллиардами документов~\cite{mdpi_fingerprint2021}.

\section{Лексико-статистические методы и метрики схожести}

\subsection{Численный признак TF $\cdot$ IDF и векторное пространство}

Для решения задачи сравнения документов на основе терминов используются лексико-статистические методы. Одним из наиболее важных методов является представление документа в виде вектора численных признаков. TF $\cdot$ IDF (Term Frequency-Inverse Document Frequency) --- это численный признак, характеризующий важность термина в документе относительно всей коллекции. Величина TF $\cdot$ IDF рассчитывается как произведение двух компонент: частоты термина в документе (TF) и его редкости в коллекции (IDF).

Построение вектора TF $\cdot$ IDF для документа происходит следующим образом: строится частотный словарь документа; для каждого слова вычисляется произведение TF $\cdot$ IDF; вектор упорядочивается по убыванию этого произведения; выбираются топ-$N$ слов с наибольшими весами и сцепляются в алфавитном порядке; в качестве сигнатуры документа вычисляется контрольная сумма (например, CRC32 или MD5) полученной строки.

Два текста сравниваются как векторы в многомерном пространстве, где каждое слово представляет одну координату. Схожесть между двумя такими векторами оценивается с помощью косинусной меры близости (косинус угла между векторами). Косинусная мера близости принимает значения от $-1$ до $+1$, где значение 1 означает полное совпадение, 0 означает ортогональность (отсутствие схожести), а $-1$ означает полную противоположность. Документы с высокой косинусной мерой близости (обычно выше порога 0.8) считаются потенциально плагиатными~\cite{tfidf_hybrid2022}.

\subsection{Метод I-Match}

Для решения задачи быстрого выявления дубликатов применяется сигнатурный подход, основанный на лексических принципах, предложенный A. Chowdhury. Основная идея метода I-Match состоит в вычислении дактилограммы для представления содержания документов на основе отобранного подмножества слов. Сначала для исходной коллекции документов строится словарь $L$, который включает слова со средними значениями IDF (исключаются очень частые служебные слова и очень редкие слова). Для каждого документа формируется множество $U$ различных слов, входящих в него, и определяется пересечение $U$ и словаря $L$. Список слов, входящих в пересечение, упорядочивается, и для него вычисляется I-Match сигнатура (хеш-функция SHA1). Два документа считаются похожими, если у них совпадают I-Match сигнатуры~\cite{zelenkov2007}.

\subsection{Метод опорных слов}

Метод опорных слов, предложенный С. Ильинским, применяется для выявления дубликатов на основе двоичного представления документа. Сначала из индекса по определенному правилу выбирается множество из $N$ слов, называемых опорными. Затем каждый документ представляется $N$-мерным двоичным вектором, где $i$-я координата равна 1, если $i$-е опорное слово имеет в документе относительную частоту выше определенного порога, и равна 0 в противном случае. Этот двоичный вектор называется сигнатурой документа. Два документа похожи, если у них совпадают сигнатуры или совпадает большинство бит. Для каждого слова строится распределение документов по внутридокументной частоте. Проводится несколько итераций оптимизации, в которых максимизируется покрытие документов при фиксированной точности, а затем максимизируется точность при фиксированном покрытии~\cite{zelenkov2007}.

\subsection{Расстояние Левенштейна}

Для решения задачи поиска близких вариантов фраз и обнаружения парафраза применяется редакционное расстояние. Расстояние Левенштейна --- это минимальное количество однозначных операций редактирования (вставка, удаление, замена символа), необходимых для преобразования одной строки в другую. Метрика полезна для поиска близких вариаций текста, включая опечатки и синтаксические ошибки. Однако расстояние Левенштейна не учитывает порядок элементов --- два текста с переставленными предложениями будут считаться различными. На больших текстах требует квадратичного времени вычисления $O(n \cdot m)$, что ограничивает его применимость в масштабных системах~\cite{mit_levenshtein2020}.

\subsection{Методы на основе предложений}

Исследование Зеленкова и Сегаловича включает алгоритмы, основанные на выборе характерных предложений документа. Эти методы решают задачу быстрого выявления потенциально похожих документов за счет анализа наиболее информативных фрагментов текста.

\textit{Long Sent}: выбираются 2 самых длинных предложения документа, сцепляются в алфавитном порядке, и вычисляется контрольная сумма CRC32. Алгоритм предполагает, что длинные предложения содержат больше информации и менее вероятно копируются с модификациями. Этот алгоритм показал наивысшую $F$-меру (0.82) среди всех исследованных методов.

\textit{Heavy Sent}: вычисляется вес каждого предложения как сумма произведений TF $\cdot$ IDF для всех слов предложения. Выбираются 2 самых тяжелых (информативных) предложения, и для них вычисляется сигнатура. Метод не учитывает порядок предложений в документе~\cite{zelenkov2007}.

\chapter{Семантические методы и стилометрический анализ}

\section{Методы семантического сопоставления}

\subsection{Встраивания слов: Word2Vec, fastText, GloVe}

Для решения задачи выявления парафраза и семантически эквивалентных текстов применяются методы семантического сопоставления. Встраивания слов (word embeddings) преобразуют слова в плотные векторы в пространстве низкой размерности, где семантически похожие слова имеют близкие представления. Word2Vec использует модель Skip-gram или CBOW для обучения на больших текстовых корпусах. fastText расширяет Word2Vec, учитывая информацию о подсловах (символьные $n$-граммы), что помогает справляться с редкими словами и опечатками. GloVe комбинирует матричную факторизацию с локальным контекстным окном. При сравнении двух документов можно усреднить встраивания всех слов в документе и получить векторное представление всего документа. Косинусная мера близости между такими усредненными векторами характеризует семантическую близость документов~\cite{arxiv_bertembed2020}.

\subsection{BERT и трансформерные модели}

BERT (Bidirectional Encoder Representations from Transformers) --- модель глубокого обучения, предварительно обученная на большом количестве текста, которая генерирует контекстные представления слов и предложений. В отличие от статических встраиваний, BERT учитывает контекст слова в предложении, что позволяет более точно захватывать смысл. Sentence-BERT (SBERT) расширяет BERT для создания встраиваний всех предложений, которые прямо оптимизированы для семантической схожести. Русскоязычные варианты BERT (например, RuBERT) обеспечивают качественное представление текстов на русском языке~\cite{aclan_phrase_bert2021}.

\subsection{Siamese и Triplet Loss архитектуры}

Siamese network состоит из двух или более копий одной и той же нейронной сети, которые обрабатывают два входа и генерируют представления, сравниваемые для определения схожести. Triplet loss минимизирует расстояние между якорным примером и похожим примером, одновременно максимизируя расстояние между якорем и непохожим примером. Эти архитектуры эффективны для обучения моделей, которые захватывают метрику сходства, необходимую для детекции плагиата~\cite{pmc_siamese2022}.

\subsection{LSTM и RNN с механизмом внимания}

Recurrent Neural Networks (RNN), особенно в форме LSTM (Long Short-Term Memory) или GRU (Gated Recurrent Unit), могут обрабатывать последовательности и захватывать долгосрочные зависимости в тексте. Добавление attention механизма позволяет модели сосредоточиться на наиболее релевантных частях входа при сравнении двух текстов. BiLSTM (bidirectional LSTM) обрабатывает текст в обоих направлениях, улучшая представление. Такие архитектуры учитывают порядок слов в документе, что критично для выявления сложных парафраз~\cite{arxiv_lstm_plag2021}.

\section{Модели авторского стиля и стилометрия}

\subsection{Основы стилометрического анализа}

Стилометрия --- это область, изучающая характеристики письменного стиля, которые отличают одного автора от другого. Предполагается, что у каждого автора есть уникальный стиль, который сохраняется даже при осознанной попытке его изменить. Признаки стилометрии включают:

\begin{itemize}

\item среднюю длину предложения, распределение длин предложений;

\item часто используемые функциональные слова (предлоги, союзы, артикли);

\item частотность части речи (POS tags);

\item лексическое разнообразие (type-token ratio);

\item использование пунктуации и заглавных букв;

\item лексическую плотность;

\item читаемость текста.

\end{itemize}

Стилометрический анализ применяется как для авторского сопоставления (определение авторства текста), так и для интринсивной детекции плагиата~\cite{acl_intrinsic_2015}.

\subsection{Интринсивная детекция плагиата}

Интринсивная (intrinsic) детекция плагиата ищет признаки копирования внутри документа, без привлечения внешних источников. Основной метод решает задачу поиска стилистических разрывов: фрагменты заимствованного текста обычно имеют стиль отличающийся от основного стиля документа. Путем анализа последовательных блоков текста можно выявить участки, где стиль резко меняется. Применяются статистические тесты для определения значимости изменения стилистических параметров. Мешок слов --- это представление текста, при котором порядок слов игнорируется, и документ представляется только набором слов с их частотами. При интринсивной детекции сравниваются мешки слов последовательных фрагментов одного документа для выявления разрывов~\cite{acl_intrinsic_2015}.

\section{Анализ и верификация цитирования}

\subsection{Проблемы и типы ошибок в цитировании}

Ошибки в цитировании принимают различные формы и являются типичными видами некорректных заимствований:

\begin{itemize}

\item некорректный источник --- ссылка приведена неправильно, источник не существует, или указывает на совершенно иное произведение;

\item отсутствующая информация в источнике --- утверждение, приписываемое источнику, в нем не содержится, или содержится в другом контексте;

\item неверная страница --- указан неправильный диапазон страниц, что затрудняет верификацию цитаты;

\item отсутствует источник --- информация используется без ссылки на источник;

\item призрачные цитаты --- источники цитируются по вторичным источникам без прямого обращения к первичному источнику.

\end{itemize}

Для решения задачи выявления этих ошибок используется автоматизированная верификация источников~\cite{cyberleninka_dissertacii}.

\subsection{Автоматическая парсинг и извлечение метаданных}

Для проверки корректности источников необходимо автоматизированно извлекать метаданные из PDF и других документов. Системы парсинга, такие как CERMINE, GROBID и PDFDataExtractor, используют компьютерное зрение и обработку естественного языка для распознавания структуры документа и извлечения текста, авторов, названия, года публикации, DOI (Digital Object Identifier), диапазонов страниц, и списка литературы. Извлеченные метаданные затем сопоставляются с библиографическими базами данных~\cite{arxiv_auto_ie2024}.

\subsection{Проверка контекста цитирования и соответствия источнику}

Анализ контекста цитирования включает проверку того, действительно ли утверждение в цитирующем тексте соответствует содержимому исходного документа. Для решения этой задачи применяются методы NLP:

\begin{itemize}

\item выделение синтаксических единиц вокруг ссылки (несколько предложений до и после цитирования);

\item извлечение соответствующих фрагментов из исходного документа, соответствующих ключевым термам из цитирующего предложения;

\item сравнение семантической схожести между контекстом ссылки и релевантными фрагментами источника с использованием косинусной меры близости;

\item оценка соответствия на основе порога схожести для определения, является ли цитирование корректным~\cite{arxiv_citecheck2025}.

\end{itemize}

\chapter{Экспериментальное исследование и оценка эффективности}

\section{Метрики оценки качества}

В качестве основных показателей качества работы алгоритмов используются полнота, точность и $F$-мера. Для оценки эффективности систем обнаружения используются следующие метрики:

\begin{itemize}

\item точность --- доля верно обнаруженных случаев плагиата из всех случаев, помеченных как плагиат. Высокая точность означает низкий процент ложных срабатываний.

\item полнота --- доля верно обнаруженных случаев из всех действительно существующих случаев плагиата. Высокая полнота означает, что система пропускает мало реальных случаев плагиата.

\item $F$-мера --- гармоническое среднее точности и полноты, дающее единую метрику качества работы алгоритма.

\end{itemize}

\section{Результаты сравнительного исследования}

Сравнение проводилось на коллекции русскоязычных веб-документов РОМИП путем поиска заимствований в текстах научных статей. Тексты включали как полные дубликаты, так и парафразы и частичные переиспользования. Система проверяла каждый алгоритм на единообразных наборах тестовых пар документов.

\begin{table}[H]

\centering

\begin{tabular}{|l|c|c|c|c|}

\hline

\textbf{Алгоритм} & \textbf{Полнота} & \textbf{Точность} & \textbf{$F$-мера} \\

\hline

Long Sent & 0.84 & 0.80 & 0.82 \\

\hline

TF & 0.60 & 0.94 & 0.73 \\

\hline

Opt Freq & 0.59 & 0.94 & 0.73 \\

\hline

TF*IDF & 0.59 & 0.95 & 0.73 \\

\hline

Heavy Sent & 0.62 & 0.86 & 0.72 \\

\hline

TF $\cdot$ IDF & 0.54 & 0.96 & 0.69 \\

\hline

Lex Rand & 0.50 & 0.97 & 0.66 \\

\hline

Descr Words & 0.44 & 0.77 & 0.56 \\

\hline

Log Shingles & 0.39 & 0.97 & 0.56 \\

\hline

Megashingles & 0.36 & 0.91 & 0.51 \\

\hline

MD5 & 0.23 & 1.00 & 0.38 \\

\hline

\end{tabular}

\caption{Сравнение метрик качества алгоритмов обнаружения заимствований в порядке убывания рейтинга}

\label{tab:algorithm_comparison}

\end{table}

Результаты показывают, что алгоритм выбора длинных предложений (Long Sent) показал наилучшие результаты по $F$-мере, сочетая высокую полноту (0.84) и точность (0.80). Лексические методы (TF, TF $\cdot$ IDF, Opt Freq) показывают высокую точность (0.94-0.97), но умеренную полноту (0.54-0.60). Метод мегашинглов уступает более простым алгоритмам, что объясняется строгими требованиями к совпадению супершинглов. Методы, основанные на шинглах и мегашинглах, хорошо подходят для масштабных систем, так как требуют минимальной памяти и времени, но их точность и полнота ниже, чем у методов на основе предложений~\cite{zelenkov2007}.

Дополнительное сравнение может быть проведено по следующим критериям: время выполнения алгоритма на документах разного размера; потребление оперативной памяти; устойчивость к переводному плагиату; обработка текстов на разных языках; качество выявления парафраза; обнаружение частичных заимствований; способность к обработке кода и структурированных данных.

